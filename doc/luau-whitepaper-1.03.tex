\documentclass{report}

\usepackage[pdftex]{hyperref}

\author{David Eklund}
\begin{document}
\title{Luau Whitepaper}

\maketitle

\begin{quote}
    Copyright \copyright  2005  David Eklund\\	
    Permission is granted to copy, distribute and/or modify this document
    under the terms of the GNU Free Documentation License, Version 1.2
    or any later version published by the Free Software Foundation;
    with no Invariant Sections, no Front-Cover Texts, and no Back-Cover Texts.
    A copy of the license is included in the section entitled "GNU
    Free Documentation License".
\end{quote}

\newpage


\tableofcontents

\newpage

\chapter{Introduction}

\section{Preface}

One of the mantras of the Open Source community is ``Release Early, Release
Often.''  This is often considered one of the central convictions of the 
``bazaar style'' of software development.  To quote Eric S. Raymond:
\begin{quote}
In the bazaar view, on the other hand, you assume that bugs are generally
shallow phenomena---or, at least, that they turn shallow pretty quickly when
exposed to a thousand eager co-developers pounding on every single new release.
Accordingly you release often in order to get more corrections, and as a
beneficial side effect you have less to lose if an occasional botch gets out
the door.
\end{quote}
There is, however, a downside to this development model.

When you install your brand-new Linux distribution, you're on the cutting edge.
You've got thousands of packages, fresh from the presses, ready to serve your
every need.  But as times wears on and you explore the inner recesses of each
of these thousands of packages, a problem arises---in your exploration, you
discover the ever-dreaded presence of bugs.  After all, no software release
is perfect---even the Big Guys have to release patches every once in a while
for their employees' minor bungles.

So what do you do?  There's a couple solutions available to you:

\begin{enumerate}
\item Live with it
\item Whine about it
\item See if a new version has been released that fixes it
\end{enumerate}

What we're most interested in is option three.  Certainly if konqueror stalls
and spits out strange characters in a foreign character set every time you try
to visit \href{http://luau.sourceforge.net}{luau.sourceforge.net} (which is, of
course, your homepage, right?), you're going to go see if the problem is
recognized and has been fixed---and let's say it has.  But now there's another
problem.

Konqueror relies on a total of \emph{35 external libraries}.  While for this issue it's likely the bug is in konqueror itself or libkonq, any small bug in any of the 34 other libraries could conceivably bring Konqueror to a grinding halt.  And if you're not the type to go source code spelunking, then you have but one choice: \emph{find and install the newest version of all 35 libraries}.  Plus Konqueror itself.

Of course, it's just good system administration in general to keep on top of all
the latest software updates---or at least the ones that can affect your system's
performance, stability, and security.  The problem with trying to keep up with
all the updates for all those thousands of packages your distribution installed
has certainly been tackled before.  Debian has apt.  FreeBSD has the ports tree.
Gentoo has a ports look-alike.  And RedHat has the Red Hat Network.  And these
all work rather well for their purposes---but what I'm here to provide is an
alternate solution.

\section{Overview}

The problem with distributors providing access to all those new software updates is that, truthfully, even they can't keep up with all the latest.  If you're not big into the cutting-edge, then apt's system works pretty well, for example---but if you want the latest and greatest, you're simply going to have to go to the source and download it yourself.

Luau isn't designed so much for production status servers (in fact, I would recommend against it) where everything needs to work together perfectly---rather, it's designed for the enthusiast who wants instantaneous updates, straight from the source.  It's for those of us who want to keep up with the latest developments for all our favorite software packages---and maybe some of the more mundane ones as well.  In the end, it's for the home user, not the corporate user.

So finally, you ask, what is Luau?  Luau is a system for retrieving software updates and status messages from package maintainers \emph{individually}---that is, it gets the information straight from the source as opposed to a third party, such as Red Hat.

Of course, there's a caveat to this: in order to get information directly from the source, the source has to provide some information (all you developers out there---it's really easy!  Go skip down to the Developer's Guide and read the overview).  So if the maintainer for your favorite software package doesn't support Luau and you want to use it, go bug her---she most likely simply hasn't heard about it.

Luau works through the use of a simple ``software respoitory'' file that's publicly available (published by the software maintainer) and describes all the currently available updates to all current software releases.  In general, this will simply consist of a list of all the currently available releases, though more complicated scenarios (along with the possibility to send messages and other configuration updates to the user-base) are also available.

For more information on this, skip on down to the Developer's Guide.

\section{Why Luau?}

At this point, you may be asking why all this trouble to implement a solution to a problem that's already been solved by the likes of apt, RHN, and so on.  As mentioned before, Luau is not meant to replace these systems unilaterally, but rather to supplement them.  Both ways of working have their pros and cons, some of which I've tried to enumerate in table \ref{table:pros}.

\begin{table}
\caption{Luau Pros and Cons}
\label{table:pros}
\begin{tabular}[htb]{|p{2cm}|p{4cm}|p{4cm}|}
\hline
 & \emph{Pros} & \emph{Cons} \\
\hline
\begin{flushleft}
Apt, RHN, ports, etc.
\end{flushleft}
& 
\begin{flushleft}
Software updates more likely to work with one another

More packages available (for the moment \ldots)

Usually more stable software
\end{flushleft}
&
\begin{flushleft}
Updates not as current

In some cases (RHN) is not free

Sometimes hard to convince to support lesser known packages

No personal connection with actual software developers
\end{flushleft}
\\
\hline
\begin{flushleft}
Luau
\end{flushleft}
&
\begin{flushleft}
Provides not only software updates but also status messages

Instant updates as soon as the maintainer posts them

Decentralized, so you don't have to rely on only one source

Only the packages you care about are monitored

If you can only convince the maintainer to support Luau, any package you like will be supported
\end{flushleft}
&
\begin{flushleft}
Software may be unstable, or may not work well together with other packages

Does not provide automatic dependency tracking and downloading (possible in future?)

Only available for packages for which respository files are provided for by the maintainer
\end{flushleft}
\\
\hline
\end{tabular}
\end{table}

\section{Revision History}

\emph{Version 1.03}\\
\emph{21 Jun. 2005}\\
This release has been converted to \LaTeX{} (rather than OpenOffice.org).  There have also been several minor updates in the text to reflect changes in Luau (for example, changing the date format from MM/DD/YYYY to YYYY-MM-DD).
\\ \\
\emph{Version 1.02}\\
\emph{14 Mar. 2004}\\
This release is updated to cover the changes in the XML format.  Also the ``Release Date Compatibility Scheme'' section has been dropped (it was obsolete) and replaced with a description of Luau's version parsing capabilities.  Section \ref{sec:luau-register} (``Using \verb+luau-register+'') has been amended to describe Autopackage's ``root name'' scheme.  Finally, some minor spelling/grammatical updates have been made.
\\ \\
\emph{Version 1.01}\\
\emph{25 Dec. 2003}\\
This release is updated to reflect the change in the updates file style and the new options added to the Luau console clients (\verb+luau+ and \verb+luau-register+).
\\ \\
\emph{Version 1.00}\\
\emph{14 Sept. 2003}\\
This is the first public release of this document.  Nothing to say here!

\section{Contact Info}

David Eklund\\
Project Maintainer and Head Developer\\
\href{mailto:deklund@fastmail.fm}{$<$deklund@fastmail.fm$>$}

Please contact me with any ideas, criticism, or bugs you have to share with me.  Furthermore, if you believe you have anything you can contribute to Luau in the form of coding, translations, web page design, or whatever else you can think of, talk to me---most likely I can use your services.

Luau Web Site: \href{http://luau.sourceforge.net}{luau.sourceforge.net}

\section{Distribution Policy}

This document is distributed under the GNU Free Documentation License.  For a full copy of the license, see Appendix A (page \pageref{sec:appendixa}).

\chapter{End-User's Guide}

\section{Luau Basics}
\label{sec:basics}

Luau is a framework for software distribution and update transmission.  The first thing to make clear is that \emph{Luau is not in any way a package manager, nor does it have anything to do with package management}.  Luau does support various types of packages---specifically RPMs, DEBs, Autopackage files, self-extracting executables, and generic tar balls---and, for those that it can, it supports automatic installation (everything except tar balls, which obviously cannot be installed automatically), but it is not designed for package installation, uninstallation, or management.  You will have to find a different tool for this---there are many available.

The second thing to make clear is that \emph{only software packages that explicitly support Luau can be updated through Luau}.  At the moment, this is a very small number (only one at the time of this writing---that would be Luau itself), but hopefully this number will increase.  If you would like to use Luau with some of the software packages you use often, contact the developers!  If enough people are interested, they will most likely start supporting it, since it takes very little upkeep on their part.

If you're particularly motivated, you can even support it yourself---you don't have to edit any source code.  You only have to provide an file describing all the current software revisions available (see the Developer's Guide for details), and then you can contact the package developers again and---unless they're extremely opposed to the idea---they will simply link up to your page.  It's all very simple; again, see the Developer's Guide a few pages down for details.

There are three basic kinds of updates supported by Luau: software updates, message updates, and Luau configuration updates.

The first kind, software updates, are the most prevalent and the main focus---they provide updates to installed software, such as when a software project makes available a new release.

The second, message updates, are for when software maintainers have an important message to distribute to their user-base.  This may be to inform users when there is a major security flaw in certain versions of the software package, or if the development team is in need of developers or artists, for example.

The final kind of update is an update to the internal configuration of Luau itself.  Luau fetches updates about software projects by downloading a configuration file specified upon installation by the software maintainer.  If this location should have to change, then this kind of update will inform you of the change in server status and---if you accept it, which you certainly should---will automatically update Luau's internal database to reflect the change.  These should be rare, but be certain to pay close attention should one become available---if you ignore or don't accept it, Luau will most likely stop working with respect to the specified program, and you'll have to update the program configuration by hand to get it working again.

\section{Using luau --- The Console App}

The \verb+luau+ console application is the most basic and fundamental user-oriented segment of the Luau suite.  That being said, you'll most likely want to use Luau-X for most of your needs, since it is much more intuitive and user-friendly.  However, when you don't have access to an X server, or you need something to use in a shell-based environment (e.g., for a shell script), \verb+luau+ is just as powerful and as flexible as Luau-X.

There are two basic modes \verb+luau+ supports---command-line based and interactive.  These should be familiar to anyone with a *nix background: the command-line based mode simply takes options from the command-line, interprets them, and executes the desired activity (download an update, list all registered programs, etc.), while interactive mode provides a prompt to the user into which (s)he can execute a string of commands.

The command line options are as follows:
{\footnotesize
\begin{verbatim}
Modes: (only one may be specified)
  -d, --download=ID     download update with given ID for specified program
  -e, --install=ID      install update with given ID for specified program
  -g, --getupdates      retrieve a list of updates for the specified program
  -i, --interactive     run in interactive mode [default]
  -l, --list            list all registered programs
  -h, --help            display this message

Extra Information:
  -m, --email=ADDRESS   email the results to ADDRESS, if any results at all
  -o, --output=PATH     when downloading an update, specify where to download
  -p, --program=NAME    specify a program
  -t, --type=TYPE       specify the type of update to download/install
  -q, --quiet           suppress all unnecessary output
  -v, --verbose         display more informational output

To download or install updates, both the program name and the type to download
MUST be specified.
\end{verbatim}
}
This can, of course, be seen any time by executing \verb+luau -h+ (or \verb+luau -help+).  Little extra explanation is needed here---the basic idea is that one (and only one) mode must be specified, in addition to any number of ``extra information'' options.  Note that (as should be obvious) you need to specify both a program name and an update type (\verb+-p+, \verb+-t+) in order to download or install an update.

Similarly, it doesn't make any sense to specify an update type when you're only trying to list registered programs; \verb+luau+ probably won't complain, though.  Interactive mode offers almost the same functionality:

{\footnotesize
\begin{verbatim}
List of commands:

get <PROGRAM-ID> <UPDATE-ID> [<PACKAGE-TYPE>] [<FILENAME>]
 - Retrieve a package of type PACKAGE-TYPE and download it to FILENAME 
  (if specified), or the current directory (if not).
install <PROGRAM-ID> <UPDATE-ID> [<PACKAGE-TYPE>]
 - Retrieve and install an update of type PACKAGE-TYPE.
list [<PROGRAM-ID>]
 - If PROGRAM-ID is specified, show whether it is registered.  Otherwise,
   list all registered programs.
updates [<PROGRAM-ID>]
 - Retrieve and list all updates available for PROGRAM-ID (or all
   registered programs, if not specified).
quit
 - Quit luau and return to command prompt

Note that for PACKAGE-TYPE, you must enter one of the following:
  RPM DEB SRC AUTOPKG EXEC
\end{verbatim}
}

Again, this can be viewed at any time by entering ``\verb+help+'' or ``\verb+?+'' at the interactive prompt.

One final note: although it's not specified in the small help message, if you select list mode (\verb+-l+) and specify a program name (\verb+-p+), it will output only the name and version of the specified package, and it'll only do that if a program with the name specified is registered with Luau.  This may be helpful in shell scripts (or \verb+configure+ scripts) to see if a program needed is installed.

Other details can always be found in the one definitive source: the man page (\verb+man luau+).

\section{Using Luau-X --- The X App}

This will most likely be where you do most of your work with Luau supported programs as an end-user.

Luau-X is the X version of \verb+luau+---it provides all the same functionality, but in a neat little graphical package.  Luau-X actually provides more functionality than \verb+luau+ in the fact that it shows and can modify the data associated with each registered program (see \verb+luau-register+ in the Other Apps section below for information on how to do this from the console).  However, the main focus of Luau-X is to provide an easy way to find available updates for all registered programs simultaneously, and to view and install those you find appropriate.

The main factor in getting used to Luau-X is simply to use it---it's a rather simple interface and finding out how to do what you want to do should be a rather simple process.  There are a few caveats to point out, however.

First of all, as you may know if you've browsed the Programmer's Guide, Luau (or rather the library interface, \verb+libuau+) provides a standard error output facility.  In Luau-X this is redirected into an error log.  To view the error log, simply select \verb+View->Error Log+ from the main window.  This may seem like a trivial fact to point out here, but it is of utmost importance in ironing out the problems that will almost certainly arise as developer's grow accustomed to the Luau interface and as the Luau system itself matures.

If you have any problems concerning downloading updates, installing updates, or anything else, this is the first place to look---it may help you solve your problem, and if not, it will be the first thing any of the Luau developers will ask for if you ask them for help.

Another thing to note is that, like the rest of Luau, Luau-X is a work-in-progress, and bugs are unfortunately to be expected.  If you have any problems, don't just live with it---email me!  First of all you should make sure you have the latest revision installed (which shouldn't be hard to find out, considering you have Luau installed and that's exactly what it's for), and if possible see if you can fix the problem on your end, but if nothing works, please feel free to drop me a line.  My e-mail address is \href{mailto:deklund@fastmail.fm}{$<$deklund@fastmail.fm$>$} and I'm always more than willing to answer questions and receive bug reports.

\section{Other Apps}

Currently, the only other application provided with the Luau suite is \verb+luau-register+.  In general, end-users shouldn't have to deal with this program---in fact, it's generally a bad idea to at all, unless you know exactly what you're doing.

\verb+luau-register+ is used to register new programs with the Luau database or to update existing programs to reflect configuration changes.  As such, it will most often be called from installation scripts run by Luau-supporting software packages as they register themselves with the Luau database in order for you to be able to check and download updates for them.

If you'd like more information about \verb+luau-register+ or how this process works as a whole, skip on down to the Developer's Guide---there's plenty of useful information on everything you need or want to know there.

\section{Final Notes}

I've already mentioned it, but I'd just like to re-emphasize---please, please contact me with all bugs, inconsistencies, or just general comments that you have.  One of the driving factors of Open Source software is the close link and communication between users and developers (and I believe Luau will only help encourage this, through the use of message updates); and as such, it is your responsibility as a user to keep the development team updated with any problems there may be.  And believe me: if you're trying out one of these first Luau releases, issues will inevitably arise.

So once again, in case you missed it, my email address is \href{mailto:deklund@fastmail.fm}{$<$deklund@fastmail.fm$>$}, and I would love to hear from you.  All correspondence is greatly appreciated, and I will try to get back to you as promptly as possible.

\chapter{Developer's Guide}

\section{How Luau Works}

The heart of Luau's updating system lies within the software repository file.  This is a file hosted somewhere public by the project maintainers which in essence describes all currently available updates and messages.

As a developer, the only thing you have to do to support Luau is to host this file and keep it updated.  Then you simply add a command to the installation script for your project (more on this later) in order to register the location of the updates file with the Luau database stored on your client's computer, and you're done.

If you want more flexibility, you can also interface with the libuau library and take direct control over the Luau facilities (see the Programmer's Guide for details), but for most purposes, simply registering your program and then letting the Luau application suite deal with the rest should be enough.

\section{Understanding the Luau XML Repository File}

The style of the updates file is a simple XML format:

\texttt{
$<$\textbf{keyword} [\emph{att1}=$<$\emph{value1}$>$} [\emph{att2}=$<$\emph{value2}$>$, \ldots]]$>$

Since there are three separate types of updates---software updates, messages, and Luau configuration updates (see section \ref{sec:basics})---it should make sense that there is a different set of keywords available to each type.  However, there's also a set of keywords common to all types: we'll cover those first.

We'll begin with an example updates file (this is in fact a copy of the \verb+example.respository.xml+ file distributed under \verb+docs/+ with Luau):

{\footnotesize
\begin{verbatim}
<?xml version="1.0"?>
<!DOCTYPE luau-repository SYSTEM 
                  "http://luau.sourceforge.net/luau-repository-1.0.dtd">

<luau-repository interface="1.0" program-id="example">
  <software version="1.1.1">
    <date>04/12/2003</date>
    <keyword>UNSTABLE</keyword>
    <short>Upgrade to myproject version 1.1.1</short>
    
    <long>
Version 1.1.1 supports CoolStuff extensions and drops legacy support
  for pre-1.0 versions.
Many bug fixes have also been made, including:
  o Preventing myproject from crashing when user inputs curse words
  o No more gamma radiation problems (hopefully)
!!! NOTE !!!
INSTALLING version 1.1.1 WILL DROP COMPATIBILITY WITH BINARIES
  COMPILED WITH PRE-1.0 versions!
If this is a problem, please stick with the 1.0 branch
    </long>
    
    <package type="RPM" 
             size="1200144"
             md5="29168bed9cc7e04652e82bc09b3c7b98">
        ftp://ftp.example.com/pub/myproject-1.1.1.rpm
    </package>
    <package type="DEB" 
             size="1432534"
             md5="98336d9f8cb04bc909a9a2a13a9e893c">
        ftp://ftp.example.com/pub/myproject-1.1.1.deb
    </package>
    <package type="SRC" 
             size="6323511"
             md5="717ab9d5dfcb8900ca14182dc0a0dbb2">
        <mirror weight="80%">
          ftp://ftp.example.com/pub/myproject-1.1.1.tar.gz
        </mirror>
        <mirror weight="20%">
          http://www.example.com/myproject/myproject-1.1.1.tar.gz
        </mirror>
    </package>
    
    <valid type="date" from="02/05/2003" to="05/15/2003"/>
  </software>
  
  <update type="message">
    <id>201</id>
    <date>04/14/2003</date>
    <short>Artists needed!</short>
    <long>
myproject is looking for some talented artists to help with the newly planned
action simulation adventure game, "MyProject: The GAME."  If you're interested,
please contact jimbo@example.com
    </long>
  </update>
  
  <update type="luau-config">
    <id>301</id>
    <date>04/14/2003</date>
    <keyword>IMPORTANT</keyword>
    <short>New server</short>
    <long>
We've got our own ftp server now, so we're trying to get everyone to switch over
to using the example.com database.  Please accept this update, since the old ftp
server will be unavailable in a month or so, and trying to autoupdate from it
after that will fail.  Thanks. :)
    </long>
    <set url="ftp://ftp.example.com/pub/myproject/updates"/>
  </update>
</luau-repository>
\end{verbatim}
}

\subsection{General Tags}

\subsubsection{luau-repository}

\texttt{
$<$\textbf{luau-repository} interface="\emph{version}" program-id="\emph{id}"$>$ \newline
\qquad [\ldots]\\
$<$/luau-repository$>$
}
\begin{itemize}
\item Values for \texttt{version}: ``1.0'', ``1.1'', ``1.2'' (\emph{required})
\item Values for \texttt{program-id}: any valid program identifier (\emph{required})
\end{itemize}

Enclosing tag for all software and updates.  The ``interface'' attribute defines the luau-repository file format version---the most current is ``1.2'' and should be specified as such in all Luau XML files.  The ``program-id'' attribute specifies the ID of the program this file is for---it is not used explicitly by Luau but specifies a context in which to read the file.

\subsubsection{software}
\texttt{
$<$\textbf{software} version="\emph{revision}"$>$[\ldots]$<$/software$>$
}
\begin{itemize}
\item Values for \texttt{version}: parse-able program version identifier (\emph{required})
\end{itemize}

Example: 

\begin{verbatim}
<software version="1.1.1">
    [...]
</software>
\end{verbatim}

Starting declaration for a software package definition.  The ``version'' attribute must be in a form Luau can parse---if your versioning scheme does not fit such a model (almost all do), then you'll have to specify that definition in a \texttt{$<$display-version$>$} tag and use a different easily-readable version identifier here.  See section \ref{sec:versionparsing} below.

\subsubsection{update}
\texttt{
$<$update type="update-type"$>$[...]$<$/update$>$
}
\begin{itemize}
\item Values for \texttt{type}:  \texttt{message}, \texttt{luau-config} (\emph{required})
\end{itemize}

Example: 

\begin{verbatim}
<update type="message">
    [...]
</update>
\end{verbatim}

Starting declaration for a new (non-software) update definition---both messages and configuration updates.

\subsubsection{id}
\texttt{$<$id$>$identifier$<$/id$>$}

Example:

\begin{verbatim}
<id>201</id>
\end{verbatim}

Assigns an ID to an update.  Required for all \texttt{message} and \texttt{luau-config} updates.  \texttt{identifier} may contain any alphanumeric characters.  Also, it must be unique (for the program in question).  There are no other specific requirements for identifier, but it suggested that you make it descriptive and concise (less than 12 characters or so if possible).

\subsubsection{date}
\texttt{$<$date$>$YYYY-MM-DD$<$/date$>$}

Example:

\begin{verbatim}
<date>2003-03-17</date>
\end{verbatim}

Assigns a date to an update.  Is valid for all three kinds of updates (and also recommended for all three), but not necessary for any.  This is not used directly by Luau in any way, but it is useful for end-users to see when an update was released.

\subsubsection{short}
\texttt{$<$short$>$short-description$<$/short$>$}

Example:

\begin{verbatim}
<short>Upgrade to the myproject version 1.1.1</short>
\end{verbatim}

Provides a short description for the update in question.  Is valid and highly recommended for all three types of updates, though not required by any.  Must be no more than one line long.

\subsubsection{long}
\texttt{$<$long$>$long-description$<$/long$>$}

Example:

\begin{verbatim}
<long>
    Version 1.1.1 supports CoolStuff extensions and drops
    legacy support for...
</long>+\\
\end{verbatim}

Provides a long description for the update in question.  Is valid and highly recommended for all three types of updates, though only required for message updates. Should be a paragraph or two long, depending.

\subsubsection{keyword}
\texttt{$<$keyword$>$keyword-name$<$/keyword$>$}

\begin{itemize}
\item Values for \texttt{keyword}: \texttt{STABLE}, \texttt{UNSTABLE}, \texttt{IMPORTANT}, [\emph{user\_defined}]
\end{itemize}

Example:

\begin{verbatim}
<keyword>STABLE</keyword>
\end{verbatim}

Assigns a keyword to an update.  This is used internally to demarcate and/or emphasize certain aspects of specific updates.  Also, if you plan on using libuau, you may define your own keywords and use them in your own program---\texttt{luau} and \texttt{luau-x} will simply ignore them.  Note that keywords are case-sensitive and all-uppercase keywords are reserved for Luau's use---if you want to use your own, make it lower case (or mixed case).

\subsection{Software Update Tags}

\subsubsection{package}

\begin{verbatim}
<package type="pkg-type" size="pkg-size" md5="md5sum">
  pkg-location OR [<mirror>...]
</package>
\end{verbatim}

\begin{itemize}
\item Values for \texttt{type}: \texttt{RPM}, \texttt{DEB}, \texttt{AUTOPKG}, \texttt{EXEC}, \texttt{SRC} (\emph{required})
\item Values for \texttt{size}: integer specifying package size in bytes (\emph{optional})
\item Values for \texttt{md5}: 32-character string specifying the md5sum of the package (\emph{optional})
\end{itemize}

Example:


{\small
\begin{verbatim}
<package type="RPM" size="1002453"
         md5sum="95dbc2411471aab385ad4adf18a8b5b7">
   <mirror weight="80%">
ftp://ftp.example.com/pub/binaries/myproject-1.1.1.rpm
   </mirror>

   <mirror weight="20%">
ftp://ftp.other.com/pub/binaries/myproject-1.1.1.rpm
   </mirror>
</package>
\end{verbatim}
}


Specify a package where this update can be downloaded.  \texttt{pkg-location} must be a URL that can be read and parsed by libcurl (any FTP or HTTP location should be fine). Required for any software update.

\subsubsection{mirror}
\verb+<mirror weight="weight%">mirror-location</mirror>+

\begin{itemize}
\item Values for \texttt{weight}: integer between 1 and 100 specifying how frequently that mirror should be chosen over the others (e.g., a mirror with weight 20\% will be picked 20\% of the time) (\emph{optional})
\end{itemize}

Example:

\begin{verbatim}
<mirror weight="80%">
    ftp://ftp.example.com/pub/binaries/myproject-1.1.1.rpm
</mirror>
\end{verbatim}

Specifies a mirror where the file can be downloaded with a specific weight.  Used when the package can be question can be obtained from several locations.  If the weight is not specified, then every mirror is given equal weight.

\subsubsection{valid}
\texttt{$<$valid type="valid-type" from="from" to="to" for="for" /$>$}

\begin{itemize}
\item Values for \texttt{type}: \texttt{date}, \texttt{version}
\end{itemize}

Example:

\begin{verbatim}
<valid type="date" from="2003-02-05" to="2003-05-15" />
<valid type="version" from="0.2.0" />
\end{verbatim}

Specifies certain conditions for which this update is valid.  You shouldn't have to use this under most cases, unless you specifically want to hide certain updates from certain users (one good example would be to put out a message only for users using a release with a major security vulnerability---in this case, you'd use something like \verb+<valid type="version" for="0.2.4" />+.  See section \ref{sec:versionparsing} below for more information on Luau's version parsing abilities.

\subsection{Message Update Tags}

There are no XML tags specific to message updates.  The \verb+<long>+ and \verb+<short>+ tags provide you with everything you should need.

\subsection{Luau Configuration Update Tags}

\subsubsection{set}
\verb+<set url="valid_url" />+

Example:

\begin{verbatim}
<set url="ftp://ftp.example.com/pub/myproject/updates" />
\end{verbatim}

Defines the new place where an updates file can be accessed.  \texttt{valid\_url} must be (surprise, surprise) a valid URL---specifically, one that can be parsed and downloaded by cURL.  At the moment, this is the only update you can make through a ``luau-config'' update.

\section{Version Parsing in Luau}
\label{sec:versionparsing}

One of the boons of the Luau style of software distribution is that any number of software revisions are available at any point in time---meaning that if you ever have the need to ``downgrade'' software due to unexpected limitations or bugs in a newer version, you always have that capability (for this reason it is suggested that distributors not remove old software revisions from the Luau XML file unless they truly are ancient).

On the end-user's side, however, one doesn't want to be bogged down by a list of old and essentially irrelevant software revisions when one only wants to check if any new releases have come out.  For this reason, both the Luau front-ends by default filter out and hide all available software versions older than the currently installed version.

The problem with this scheme is that not all versioning schemes fit the same template.  Fortunately Luau's version parsing algorithm (based off the one used by Autopackage) is fairly flexible, but there will always be some exceptions.

In general, any versioning scheme that roughly follows any of the following formats shouldn't have a problem:

\begin{description}
\item[1.2.4, 3.5-rc10, 3:1:2]--- Any scheme that follows the pattern a.b.c.d.(\ldots), where the first number (or letter) represents the most ``significant'' release number (e.g., 3.2 $>$ 2.3) will work.  Any non-alphanumeric character can be used to separate the major, minor, etc. revisions.

Note that an alphabetic release always precedes a numeric release---for example, 1.2 $>$ 1.2b, 2.0 $>$ 2.0-rc10, 2.1.2 $>$ 2.1b.  This may cause some problems for some people---see below.

\item[1999-12-03, 20030115]--- These date-like versions will work only if the year is listed first, followed by the month, followed by the day (i.e., with the most ``significant'' index first).  It should be clear to see that this is parsed just like the major.minor.patch case, except with different separators.
\end{description}

These, on the other hand, will not work:

\begin{description}
\item[1.0 $<$ 1.0a $<$ 1.0b $<$ 1.1]--- As mentioned before, an alphabetic release is always ``less than'' a numeric release: in this case, version ``1.0'' will be read as being a later release than ``1.0a.''  There isn't really anything I can do about this, because there is an intrinsic ambiguity in the notation---when you say ``1.0a'', do you mean ``1.0 alpha'' or ``supplement to 1.0?''  More schemes fall into the former case than the latter, so that's what gets supported.

\item[``Decreasing'' Version Numbers] (e.g., ``counting down from 100'')--- I've seen a few people get really creative with their versioning scheme by doing things like counting down from a certain original release number (like 100 or 1000), decrementing the version number with each new release (100, 99, 98, and so on\ldots apparently, by the time you get to zero, you will have obtained the perfect release candidate).  This can't be supported for obvious reasons.

\item[12/24/2003, 01-03-2004]--- These schemes are not supported because they don't list the requisite parts of the version in a ``logical'' ordering.

While it would be possible to try to get Luau to understand these schemes, it would almost inevitably break certain other non-date based valid schemes---again, there's an ambiguity in a version like ``1.2.98'': do you mean ``January 2nd, 1998'' or ``Major: 1, Minor: 2, Patch: 98?''  In the former case, ``1.2.99'' would certainly be a later release than ``2.2.98,'' but the opposite is true in the latter.

If you want to use a release date scheme, please use one as listed above or read on to find out how to get it to work.
\end{description}

Most people shouldn't be affected by this (from browsing freshmeat.net, almost all projects are labeled in a ``major.minor.patch'' or ``major.minor'' form), but if your project's versioning scheme causes a problem, there is something you can do---though it isn't particularly ideal.

What you have to do is use a different versioning scheme when you're amending your Luau XML file---I would suggest using a release date scheme as mentioned above to prevent any kind of confusion---and then put the ``real'' version number in a \verb+<display-version>+ tag in the relevant \verb+<software>+ section.  

For example, say you were using a decreasing scheme as mentioned above, and on February 13th, 2004 you released version ``58.''  Then you may do something as follows:

\begin{verbatim}
<software version="2004-02-13">
	<display-version>58</display-version>
	...
</software>
\end{verbatim}

If on the other hand you used a ``1.0 < 1.0a < 1.0b < 1.1'' scheme, something as follows may work better (say you were releasing version ``1.0b''):

\begin{verbatim}
<software version="1.0.2">
	<display-version>1.0b</display-version>
	...
</software>
\end{verbatim}

\ldots where the alphabetic character ``b'' has been replaced with ``.2''.  This would work as you would expect.

Having said all that, however, I highly suggest that you stick to a ``standard'' versioning scheme.  Doing otherwise is only bound to confuse your users.

\section{Using luau-register}
\label{sec:luau-register}

\texttt{luau-register} is an incredibly simple application, but it's also a very important one.  Since Luau works on a decentralized basis, it's your responsibility as a developer to register your program with the internal database on each of your user's systems

This is a simpler task than it sounds---all you have to do is, after your software has been installed (in a post-install script, for example), automatically call luau-register with the appropriate values---program ID, version, release date, etc.

Luau's command arguments are as follows (as usual, you can also find this out by simply running \texttt{luau-register -h}):

{\footnotesize
\begin{verbatim}
Usage: luau-register [OPTION] ... program_id

Register a program with the specified details in the Luau database.

Options:
  -r, --remove                     remove specified program from the database

  -u, --url=LOCATION               location of Luau software repository file
  -d, --date="MM/DD/YYYY"          release date of this program version
  -k, --keywords="KEY1,KEY2,..."   keywords for this program revision
  -v, --version=VERSION            installed version of this program
  -i, --interface=VERSION          interface version for this program
  -n, --shortname=NAME             short "UNIX name" of program
  -f, --fullname=NAME              full/display name of program
  -s, --desc=DESC                  one-line description of program

  -l, --from-url=URL               read program information from specified URL
  -e, --from-file=FILE             read program information from local file

  -h, --help                       display this help message

  Note that both the server and the software repository URL must be set for Luau
  to work for any given program.  If no short name is specified, the program_id
  is used.  Please see the Luau whitepaper for more information.
\end{verbatim}
}

An important thing to note is that your program ID must be unique from all other ID's used by other software projects that use Luau.  The best way to deal with this is to use Autopackage's ``root name'' naming scheme, which looks something like this:

\begin{verbatim}
	@example.org/program-name
\end{verbatim}

\ldots where ``example.org'' is the domain owned by the software maintainer, and \texttt{program-id} is just the name of the program in question.  This way all program IDs are made unique and collisions should be very rare since the root name system leverages off the uniqueness of top level domains.

These names will never be exposed to the user, who will instead always see either the ``short name'' (specified by \texttt{luau-register}) in the context of the command line, or the ``full name'' (again specified by \texttt{luau-register}) in the context of the graphical environment.  As an example, Luau's root name in this scheme is ``\verb+@luau.sf.net/luau+''; its short name is ``luau''; and its full name is ``Luau Software Updater.''

\section{Good Policies (or, \emph{How Not To Annoy Your Users})}
\label{sec:goodpolicies}

In general, the really important thing to realize when using Luau is that your users are using it to retrieve updates for the project, and that's it.  Luau provides an avenue for you to send messages to your user-base (a feature that you'll hopefully find useful), but it's important not to abuse it.

Users want to hear about important security updates, or possibly occasional calls for recruitment---they don't, on the other hand, want to hear about your personal life, or other things unrelated to the software at hand.  Don't, for example, use it as your personal blog (unless a desire to read your blog was the specific reason people downloaded your software---unlikely, at best).

Overall, message updates should be relatively rare things.  If you're putting them out more than about once a month, maybe you should rethink what you consider to be truly important.

Particularly, don't send out a message update with every new software release.  This is what software updates are for---no reason to be redundant.  Only put out a message announcing a software release if you think it's \emph{really} important---if, for example, it fixes a major security hole, or if it's a major enhancement over the previous versions (and even that's questionable).

\chapter{Programmer's Guide}

Sorry, this hasn't been written yet!  However, there is a rather thorough description of the libuau interface (made using doxygen) available both in the Luau distribution (under docs/api/) and online: \href{http://luau.sourceforge.net/api/}{luau.sourceforge.net/api/}.

Specifically, you'll want to look at the \texttt{libuau.h} interface, since this is the only methods you, as a \texttt{libuau} programmer, will have access too.  The interface is relatively simple, but beware---in its current state, it is very volatile.  There will most likely be many updates and changes before the interface stabilizes---therefore, I recommend that, unless you specifically need to extra functionality, you simply stick to the means available (ie, \texttt{luau} and \texttt{luau-x}) instead of writing your own solution.

\chapter{Contributor's Guide}

Sorry, this hasn't been written yet either!  Once again, however, there is a detailed description of the internal API as well as the \texttt{libuau} interface in the Luau distribution (under \texttt{docs/api/}) and online: \href{http://luau.sourceforge.net/api/}{luau.sourceforge.net/api/}.

Also, I have attempted to comment the source code and make it as clear as possible for anyone interested in contributing.  As always, if you have any questions or comments (or, even better, patches!), please e-mail me at <\href{mailto:deklund@fastmail.fm}{deklund@fastmail.fm}>.  Thank you for your help!

\chapter{Conclusion}

As a developer, I have invested a number of months into the preparation, planning, execution, and documentation of the Luau software suite.  As such, it is my sincere hope that it will be found useful and applicable to the Open Source community.

However, I realize that my work alone cannot fulfill the needs of every software developer, and that certainly the needs I have addressed with the creation of Luau will not necessarily meet all the needs of others.  Therefore, I ask once again for anyone who has taken the time to read through this document, or even anyone who only read the Introduction and then just skimmed the rest (you know who you are) to e-mail me any feedback you may have.  I'm open to suggestions, criticism, and any offers of help you can provide.

Also let me reiterate that I am always open to providing support---if you're a developer who wants to implement Luau in your software project, but are having problems getting it to work or can't understand one of the specifics, don't give up: just ask me!

I have tried to make everything as clear as possible both in coding and in documentation, but I'm no expert technical writer---I'm sure there have been spots in this document where things got confusing, redundant, or simply irrelevant.  As always, any clarification you need I will happily provide.  Once more, my e-mail address is <\href{mailto:deklund@fastmail.fm}{deklund@fastmail.fm}>---take advantage of it.

Finally, thank you for taking the time to find out more about Luau.  I believe that it has the capability of being a boon to the Open Source development model---but only if it is widely heard about and implemented by many software projects.  If you know anyone who may be interested in using it, please feel free to tell them about it and redirect them to this document and the Luau web-page, \url{http://luau.sourceforge.net}.

---\emph{David Eklund, Project Maintainer and Head Developer}

\appendix

\chapter{GNU Free Documentation License}
\label{sec:appendixa}
 \begin{center}

       Version 1.2, November 2002


 Copyright \copyright 2000,2001,2002  Free Software Foundation, Inc.
 
 \bigskip
 
     59 Temple Place, Suite 330, Boston, MA  02111-1307  USA
  
 \bigskip
 
 Everyone is permitted to copy and distribute verbatim copies
 of this license document, but changing it is not allowed.
\end{center}


\begin{center}
{\bf\large Preamble}
\end{center}

The purpose of this License is to make a manual, textbook, or other
functional and useful document "free" in the sense of freedom: to
assure everyone the effective freedom to copy and redistribute it,
with or without modifying it, either commercially or noncommercially.
Secondarily, this License preserves for the author and publisher a way
to get credit for their work, while not being considered responsible
for modifications made by others.

This License is a kind of "copyleft", which means that derivative
works of the document must themselves be free in the same sense.  It
complements the GNU General Public License, which is a copyleft
license designed for free software.

We have designed this License in order to use it for manuals for free
software, because free software needs free documentation: a free
program should come with manuals providing the same freedoms that the
software does.  But this License is not limited to software manuals;
it can be used for any textual work, regardless of subject matter or
whether it is published as a printed book.  We recommend this License
principally for works whose purpose is instruction or reference.


\begin{center}
{\Large\bf 1. APPLICABILITY AND DEFINITIONS}
\addcontentsline{toc}{section}{1. APPLICABILITY AND DEFINITIONS}
\end{center}

This License applies to any manual or other work, in any medium, that
contains a notice placed by the copyright holder saying it can be
distributed under the terms of this License.  Such a notice grants a
world-wide, royalty-free license, unlimited in duration, to use that
work under the conditions stated herein.  The \textbf{"Document"}, below,
refers to any such manual or work.  Any member of the public is a
licensee, and is addressed as \textbf{"you"}.  You accept the license if you
copy, modify or distribute the work in a way requiring permission
under copyright law.

A \textbf{"Modified Version"} of the Document means any work containing the
Document or a portion of it, either copied verbatim, or with
modifications and/or translated into another language.

A \textbf{"Secondary Section"} is a named appendix or a front-matter section of
the Document that deals exclusively with the relationship of the
publishers or authors of the Document to the Document's overall subject
(or to related matters) and contains nothing that could fall directly
within that overall subject.  (Thus, if the Document is in part a
textbook of mathematics, a Secondary Section may not explain any
mathematics.)  The relationship could be a matter of historical
connection with the subject or with related matters, or of legal,
commercial, philosophical, ethical or political position regarding
them.

The \textbf{"Invariant Sections"} are certain Secondary Sections whose titles
are designated, as being those of Invariant Sections, in the notice
that says that the Document is released under this License.  If a
section does not fit the above definition of Secondary then it is not
allowed to be designated as Invariant.  The Document may contain zero
Invariant Sections.  If the Document does not identify any Invariant
Sections then there are none.

The \textbf{"Cover Texts"} are certain short passages of text that are listed,
as Front-Cover Texts or Back-Cover Texts, in the notice that says that
the Document is released under this License.  A Front-Cover Text may
be at most 5 words, and a Back-Cover Text may be at most 25 words.

A \textbf{"Transparent"} copy of the Document means a machine-readable copy,
represented in a format whose specification is available to the
general public, that is suitable for revising the document
straightforwardly with generic text editors or (for images composed of
pixels) generic paint programs or (for drawings) some widely available
drawing editor, and that is suitable for input to text formatters or
for automatic translation to a variety of formats suitable for input
to text formatters.  A copy made in an otherwise Transparent file
format whose markup, or absence of markup, has been arranged to thwart
or discourage subsequent modification by readers is not Transparent.
An image format is not Transparent if used for any substantial amount
of text.  A copy that is not "Transparent" is called \textbf{"Opaque"}.

Examples of suitable formats for Transparent copies include plain
ASCII without markup, Texinfo input format, LaTeX input format, SGML
or XML using a publicly available DTD, and standard-conforming simple
HTML, PostScript or PDF designed for human modification.  Examples of
transparent image formats include PNG, XCF and JPG.  Opaque formats
include proprietary formats that can be read and edited only by
proprietary word processors, SGML or XML for which the DTD and/or
processing tools are not generally available, and the
machine-generated HTML, PostScript or PDF produced by some word
processors for output purposes only.

The \textbf{"Title Page"} means, for a printed book, the title page itself,
plus such following pages as are needed to hold, legibly, the material
this License requires to appear in the title page.  For works in
formats which do not have any title page as such, "Title Page" means
the text near the most prominent appearance of the work's title,
preceding the beginning of the body of the text.

A section \textbf{"Entitled XYZ"} means a named subunit of the Document whose
title either is precisely XYZ or contains XYZ in parentheses following
text that translates XYZ in another language.  (Here XYZ stands for a
specific section name mentioned below, such as \textbf{"Acknowledgements"},
\textbf{"Dedications"}, \textbf{"Endorsements"}, or \textbf{"History"}.)  
To \textbf{"Preserve the Title"}
of such a section when you modify the Document means that it remains a
section "Entitled XYZ" according to this definition.

The Document may include Warranty Disclaimers next to the notice which
states that this License applies to the Document.  These Warranty
Disclaimers are considered to be included by reference in this
License, but only as regards disclaiming warranties: any other
implication that these Warranty Disclaimers may have is void and has
no effect on the meaning of this License.


\begin{center}
{\Large\bf 2. VERBATIM COPYING}
\addcontentsline{toc}{section}{2. VERBATIM COPYING}
\end{center}

You may copy and distribute the Document in any medium, either
commercially or noncommercially, provided that this License, the
copyright notices, and the license notice saying this License applies
to the Document are reproduced in all copies, and that you add no other
conditions whatsoever to those of this License.  You may not use
technical measures to obstruct or control the reading or further
copying of the copies you make or distribute.  However, you may accept
compensation in exchange for copies.  If you distribute a large enough
number of copies you must also follow the conditions in section 3.

You may also lend copies, under the same conditions stated above, and
you may publicly display copies.


\begin{center}
{\Large\bf 3. COPYING IN QUANTITY}
\addcontentsline{toc}{section}{3. COPYING IN QUANTITY}
\end{center}


If you publish printed copies (or copies in media that commonly have
printed covers) of the Document, numbering more than 100, and the
Document's license notice requires Cover Texts, you must enclose the
copies in covers that carry, clearly and legibly, all these Cover
Texts: Front-Cover Texts on the front cover, and Back-Cover Texts on
the back cover.  Both covers must also clearly and legibly identify
you as the publisher of these copies.  The front cover must present
the full title with all words of the title equally prominent and
visible.  You may add other material on the covers in addition.
Copying with changes limited to the covers, as long as they preserve
the title of the Document and satisfy these conditions, can be treated
as verbatim copying in other respects.

If the required texts for either cover are too voluminous to fit
legibly, you should put the first ones listed (as many as fit
reasonably) on the actual cover, and continue the rest onto adjacent
pages.

If you publish or distribute Opaque copies of the Document numbering
more than 100, you must either include a machine-readable Transparent
copy along with each Opaque copy, or state in or with each Opaque copy
a computer-network location from which the general network-using
public has access to download using public-standard network protocols
a complete Transparent copy of the Document, free of added material.
If you use the latter option, you must take reasonably prudent steps,
when you begin distribution of Opaque copies in quantity, to ensure
that this Transparent copy will remain thus accessible at the stated
location until at least one year after the last time you distribute an
Opaque copy (directly or through your agents or retailers) of that
edition to the public.

It is requested, but not required, that you contact the authors of the
Document well before redistributing any large number of copies, to give
them a chance to provide you with an updated version of the Document.


\begin{center}
{\Large\bf 4. MODIFICATIONS}
\addcontentsline{toc}{section}{4. MODIFICATIONS}
\end{center}

You may copy and distribute a Modified Version of the Document under
the conditions of sections 2 and 3 above, provided that you release
the Modified Version under precisely this License, with the Modified
Version filling the role of the Document, thus licensing distribution
and modification of the Modified Version to whoever possesses a copy
of it.  In addition, you must do these things in the Modified Version:

\begin{itemize}
\item[A.] 
   Use in the Title Page (and on the covers, if any) a title distinct
   from that of the Document, and from those of previous versions
   (which should, if there were any, be listed in the History section
   of the Document).  You may use the same title as a previous version
   if the original publisher of that version gives permission.
   
\item[B.]
   List on the Title Page, as authors, one or more persons or entities
   responsible for authorship of the modifications in the Modified
   Version, together with at least five of the principal authors of the
   Document (all of its principal authors, if it has fewer than five),
   unless they release you from this requirement.
   
\item[C.]
   State on the Title page the name of the publisher of the
   Modified Version, as the publisher.
   
\item[D.]
   Preserve all the copyright notices of the Document.
   
\item[E.]
   Add an appropriate copyright notice for your modifications
   adjacent to the other copyright notices.
   
\item[F.]
   Include, immediately after the copyright notices, a license notice
   giving the public permission to use the Modified Version under the
   terms of this License, in the form shown in the Addendum below.
   
\item[G.]
   Preserve in that license notice the full lists of Invariant Sections
   and required Cover Texts given in the Document's license notice.
   
\item[H.]
   Include an unaltered copy of this License.
   
\item[I.]
   Preserve the section Entitled "History", Preserve its Title, and add
   to it an item stating at least the title, year, new authors, and
   publisher of the Modified Version as given on the Title Page.  If
   there is no section Entitled "History" in the Document, create one
   stating the title, year, authors, and publisher of the Document as
   given on its Title Page, then add an item describing the Modified
   Version as stated in the previous sentence.
   
\item[J.]
   Preserve the network location, if any, given in the Document for
   public access to a Transparent copy of the Document, and likewise
   the network locations given in the Document for previous versions
   it was based on.  These may be placed in the "History" section.
   You may omit a network location for a work that was published at
   least four years before the Document itself, or if the original
   publisher of the version it refers to gives permission.
   
\item[K.]
   For any section Entitled "Acknowledgements" or "Dedications",
   Preserve the Title of the section, and preserve in the section all
   the substance and tone of each of the contributor acknowledgements
   and/or dedications given therein.
   
\item[L.]
   Preserve all the Invariant Sections of the Document,
   unaltered in their text and in their titles.  Section numbers
   or the equivalent are not considered part of the section titles.
   
\item[M.]
   Delete any section Entitled "Endorsements".  Such a section
   may not be included in the Modified Version.
   
\item[N.]
   Do not retitle any existing section to be Entitled "Endorsements"
   or to conflict in title with any Invariant Section.
   
\item[O.]
   Preserve any Warranty Disclaimers.
\end{itemize}

If the Modified Version includes new front-matter sections or
appendices that qualify as Secondary Sections and contain no material
copied from the Document, you may at your option designate some or all
of these sections as invariant.  To do this, add their titles to the
list of Invariant Sections in the Modified Version's license notice.
These titles must be distinct from any other section titles.

You may add a section Entitled "Endorsements", provided it contains
nothing but endorsements of your Modified Version by various
parties--for example, statements of peer review or that the text has
been approved by an organization as the authoritative definition of a
standard.

You may add a passage of up to five words as a Front-Cover Text, and a
passage of up to 25 words as a Back-Cover Text, to the end of the list
of Cover Texts in the Modified Version.  Only one passage of
Front-Cover Text and one of Back-Cover Text may be added by (or
through arrangements made by) any one entity.  If the Document already
includes a cover text for the same cover, previously added by you or
by arrangement made by the same entity you are acting on behalf of,
you may not add another; but you may replace the old one, on explicit
permission from the previous publisher that added the old one.

The author(s) and publisher(s) of the Document do not by this License
give permission to use their names for publicity for or to assert or
imply endorsement of any Modified Version.


\begin{center}
{\Large\bf 5. COMBINING DOCUMENTS}
\addcontentsline{toc}{section}{5. COMBINING DOCUMENTS}
\end{center}


You may combine the Document with other documents released under this
License, under the terms defined in section 4 above for modified
versions, provided that you include in the combination all of the
Invariant Sections of all of the original documents, unmodified, and
list them all as Invariant Sections of your combined work in its
license notice, and that you preserve all their Warranty Disclaimers.

The combined work need only contain one copy of this License, and
multiple identical Invariant Sections may be replaced with a single
copy.  If there are multiple Invariant Sections with the same name but
different contents, make the title of each such section unique by
adding at the end of it, in parentheses, the name of the original
author or publisher of that section if known, or else a unique number.
Make the same adjustment to the section titles in the list of
Invariant Sections in the license notice of the combined work.

In the combination, you must combine any sections Entitled "History"
in the various original documents, forming one section Entitled
"History"; likewise combine any sections Entitled "Acknowledgements",
and any sections Entitled "Dedications".  You must delete all sections
Entitled "Endorsements".

\begin{center}
{\Large\bf 6. COLLECTIONS OF DOCUMENTS}
\addcontentsline{toc}{section}{6. COLLECTIONS OF DOCUMENTS}
\end{center}

You may make a collection consisting of the Document and other documents
released under this License, and replace the individual copies of this
License in the various documents with a single copy that is included in
the collection, provided that you follow the rules of this License for
verbatim copying of each of the documents in all other respects.

You may extract a single document from such a collection, and distribute
it individually under this License, provided you insert a copy of this
License into the extracted document, and follow this License in all
other respects regarding verbatim copying of that document.


\begin{center}
{\Large\bf 7. AGGREGATION WITH INDEPENDENT WORKS}
\addcontentsline{toc}{section}{7. AGGREGATION WITH INDEPENDENT WORKS}
\end{center}


A compilation of the Document or its derivatives with other separate
and independent documents or works, in or on a volume of a storage or
distribution medium, is called an "aggregate" if the copyright
resulting from the compilation is not used to limit the legal rights
of the compilation's users beyond what the individual works permit.
When the Document is included in an aggregate, this License does not
apply to the other works in the aggregate which are not themselves
derivative works of the Document.

If the Cover Text requirement of section 3 is applicable to these
copies of the Document, then if the Document is less than one half of
the entire aggregate, the Document's Cover Texts may be placed on
covers that bracket the Document within the aggregate, or the
electronic equivalent of covers if the Document is in electronic form.
Otherwise they must appear on printed covers that bracket the whole
aggregate.


\begin{center}
{\Large\bf 8. TRANSLATION}
\addcontentsline{toc}{section}{8. TRANSLATION}
\end{center}


Translation is considered a kind of modification, so you may
distribute translations of the Document under the terms of section 4.
Replacing Invariant Sections with translations requires special
permission from their copyright holders, but you may include
translations of some or all Invariant Sections in addition to the
original versions of these Invariant Sections.  You may include a
translation of this License, and all the license notices in the
Document, and any Warranty Disclaimers, provided that you also include
the original English version of this License and the original versions
of those notices and disclaimers.  In case of a disagreement between
the translation and the original version of this License or a notice
or disclaimer, the original version will prevail.

If a section in the Document is Entitled "Acknowledgements",
"Dedications", or "History", the requirement (section 4) to Preserve
its Title (section 1) will typically require changing the actual
title.


\begin{center}
{\Large\bf 9. TERMINATION}
\addcontentsline{toc}{section}{9. TERMINATION}
\end{center}


You may not copy, modify, sublicense, or distribute the Document except
as expressly provided for under this License.  Any other attempt to
copy, modify, sublicense or distribute the Document is void, and will
automatically terminate your rights under this License.  However,
parties who have received copies, or rights, from you under this
License will not have their licenses terminated so long as such
parties remain in full compliance.


\begin{center}
{\Large\bf 10. FUTURE REVISIONS OF THIS LICENSE}
\addcontentsline{toc}{section}{10. FUTURE REVISIONS OF THIS LICENSE}
\end{center}


The Free Software Foundation may publish new, revised versions
of the GNU Free Documentation License from time to time.  Such new
versions will be similar in spirit to the present version, but may
differ in detail to address new problems or concerns.  See
http://www.gnu.org/copyleft/.

Each version of the License is given a distinguishing version number.
If the Document specifies that a particular numbered version of this
License "or any later version" applies to it, you have the option of
following the terms and conditions either of that specified version or
of any later version that has been published (not as a draft) by the
Free Software Foundation.  If the Document does not specify a version
number of this License, you may choose any version ever published (not
as a draft) by the Free Software Foundation.


\begin{center}
{\Large\bf ADDENDUM: How to use this License for your documents}
\addcontentsline{toc}{section}{ADDENDUM: How to use this License for your documents}
\end{center}

To use this License in a document you have written, include a copy of
the License in the document and put the following copyright and
license notices just after the title page:

\bigskip
\begin{quote}
    Copyright \copyright  YEAR  YOUR NAME.
    Permission is granted to copy, distribute and/or modify this document
    under the terms of the GNU Free Documentation License, Version 1.2
    or any later version published by the Free Software Foundation;
    with no Invariant Sections, no Front-Cover Texts, and no Back-Cover Texts.
    A copy of the license is included in the section entitled "GNU
    Free Documentation License".
\end{quote}
\bigskip
    
If you have Invariant Sections, Front-Cover Texts and Back-Cover Texts,
replace the "with...Texts." line with this:

\bigskip
\begin{quote}
    with the Invariant Sections being LIST THEIR TITLES, with the
    Front-Cover Texts being LIST, and with the Back-Cover Texts being LIST.
\end{quote}
\bigskip
    
If you have Invariant Sections without Cover Texts, or some other
combination of the three, merge those two alternatives to suit the
situation.

If your document contains nontrivial examples of program code, we
recommend releasing these examples in parallel under your choice of
free software license, such as the GNU General Public License,
to permit their use in free software.


\end{document}
